\section{Conclusions}
%_________________________________________________________
%Move 1: Background information (research purposes, theory,
%methodology)
A high correlation between the aft thruster utilization (TU) and fuel consumption (FC) has been observed for trips between Landskona and Ven with the double ended ferry Uraniborg.  A controlled experiment was conducted onboard the ship, where the TU was varied to infer if the observed correlation implies a direct causal relationship. Such a relationship would mean that the TU can be used as an optimization parameter for FC reduction.

%_________________________________________________________
%Move 2: Summarizing and reporting key results. (oblig.)
The experiment was conducted with two different operators of the ship: the captain and the first mate. The captain was instructed to operate the ship in a new way: with as much aft thruster utilization as possible; the first mate was instructed to run the ship in a more normal way: using both the forward and aft thruster.
For a reliable inference, the TU needs to be the only parameter varied during the experiment. Therefore, the operator was changed every other round trip, from Landskrona to Ven and back, so that the experiments would have as similar conditions as possible, in terms of traffic, wind, waves, and currents.
There were however also differences in average ship speed during the experiments, which has a big influence on the FC. The FC data was therefore corrected, by using a FC prediction model to estimate the speed influence.

The experiment was conducted according to the plan where the captain used the aft thruster almost exclusively, obtaining an average TU of 0.99, while during the same time the mate used a more normal operation (TU=0.69). The difference in FC during the experiment between the two operators was substantial, with \savingpctexperiment \% lower FC for the captain. 
%_________________________________________________________
%Move 3: Commenting on key results (making claims, explaining the results,
%comparing the new work with previous studies, offering
%alternative explanations) (oblig.)
The experimental results were also compared with data collected from a reference period before the experiment, with the same operators running the ship on a similar schedule. The captain had a much lower average TU (0.77) during the reference period compared to the experiment, in contrast to the mate who had a similar TU during both periods. The captain had a lower FC during the reference period as well (\savingpctbeforeexperiment \%).
The reduction was however much larger during the experiment, which most likely implies that there is a direct causality between TU and FC; the increased TU, used by the captain during the experiment, seems to have caused a substantial reduction of the FC. The FC reduction from increased TU (0.77 to 0.99) is estimated to be at least \savingthrusterallocationpct \% when subtracting the reference period reduction -- between captain and mate -- from the reduction obtained at the experiment. 
% previous studies: "designing a double ended ferry"

%_________________________________________________________
%Move 4: Stating the limitations of the study
The results presented in this paper come from one experiment that was conducted during three days of relatively good weather with two operators that run the ship quite differently already from before.
A repeated experiment, during other weather conditions, and with other operators, preferably with operators who have more similar FC than the captain and mate had before the present experiment, would increase the confidence in the present results.

%_________________________________________________________
%Move 5: Making recommendations for future implementation and/or for
%future research
The captain used a slightly higher TU than the mate already during the reference period, which partly explains the captains reduced FC before the experiment. % Is this the whole explanation?
This reduction is however still partly unexplained and requires more thorough investigations that could lead to suggestions for further FC reductions of Uraniborg.