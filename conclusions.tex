\section{Conclusions}
%_________________________________________________________
%Move 1: Background information (research purposes, theory,
%methodology)
A high correlation between the aft thruster utilization (TU) and fuel consumption (FC) has been observed for trips between Landskona and Ven with the double ended ferry Uraniborg. A direct causality, where TU causes FC, means that TU could be used as an optimization parameter to reduce the FC. The observed correlation suggests that there could be such a causality, but causal inference needs to be conducted to confirm it.
An experiment was conducted onboard Uraniborg where the TU was varied to infer the direct causal relationship.

%_________________________________________________________
%Move 2: Summarizing and reporting key results. (oblig.)
The experiment was conducted with two different operators of the ship: the captain and the first mate. The captain was instructed to run the ship in a new way: with as much aft thruster utilization as possible, and the first mate was instructed to run the ship in a more normal way: using both the forward and aft thruster.
The operator was changed every other round trip, from Landskrona to Ven and back, so that the experiments with the two operators would have as similar conditions as possible, in terms of traffic, wind, waves, and currents.
The aim was that the TU would be the only parameter varied during the experiment. There were however also differences in average speed, which has a big influence on the FC. The FC data was therefore corrected, by using a FC prediction model to estimate the speed influence on FC.
The experiment was conducted according to the plan where the captain used the aft thruster almost exclusively, obtaining an average TU of 0.99, and the mate used a more normal operation: obtaining TU=0.69.
The captain achieved \savingpctexperiment \% lower FC compared to the mate during the experiment.
Data was also collected from a reference period before the experiment, where the same operators where running the ship on a similar schedule as during the experiment. The captain had a much lower average TU during the reference period (0.77), compared to the experiment in contrast to the mate who had a very similar TU during both the reference period and the experiment.
The captain achieved \savingpctbeforeexperiment \% lower FC than the mate during the reference period. 

%_________________________________________________________
%Move 3: Commenting on key results (making claims, explaining the results,
%comparing the new work with previous studies, offering
%alternative explanations) (oblig.)
The fact that the captain achieved a lower fuel consumption already during the reference period, could be partly explained by a slightly higher TU than the mate. There are probably also other things that differ between the captain's and the mate's operation of the ship, which can explain the lower FC. The FC difference was however even larger between the captain and mate during the experiment, which means that the increased TU has reduced the FC with at least \savingthrusterallocationpct \% and that a direct causal relation between TU and FC can be inferred.

%_________________________________________________________
%Move 4: Stating the limitations of the study
The fact that the captain was already better than the mate in optimizing the FC, before the experiment is a bit worrying. A repeated experiment with other operators of the ship would strengthen the conclusion even more.

%_________________________________________________________
%Move 5: Making recommendations for future implementation and/or for
%future research
The unexplained FC reduction obtained by the captain already before the experiment needs further investigations, to strengthen the belief in the results from this experiment, and also to document the captain's efficient way to operate the ship, so that it can be adopted by other crew members also. 