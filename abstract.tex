%Purpose and scope
As humans, we often think in terms of cause and effect — if we understand why something happened, we can change our behavior to improve future outcomes. The same goes for ships, if we can understand how they work, we can operate them in safer and more energy efficient ways. Model test experiments or computations is one way to gain this understanding for ship hydrodynamics. Data driven approaches on real ship operational data is another way, which can be more relevant as it concerns the real ship, and not just a model. The fact that this data is collected at sea, being an uncontrollable environment makes it often hard to determine the causation, the cause and effect relations. The correlation between variables can not tell if there is a direct causation where A causes B or the other way around in reversed causation, for instance:
\begin{quote}
do windmills generate the wind, or is it the other way around? 
\end{quote}

Both A and B can also be caused by a third hidden variable C in common causation, for instance:
\begin{quote}
if ice cream sales increase, the rate of drowning deaths also increases, so should you avoid ice cream when swimming? 
\end{quote}

Determining the cause and effect from real world data is a difficult but important task. For instance proving that the emission of green house gases causes global warming is perhaps the most important task of our time. 


%Methods

%Results

%Conclusion