% Move 1 - Background/introduction/situation
Today, there is an increased use of statistical analysis or ML on real data, with the aim to propose changes in optimizing a system's behaviour. Before such changes could be proposed, determining the cause and effect relation between the variables -- causality -- is first necessary.
The aim of standard statistical analysis or machine learning (ML) is to find trends or associations between variables; causal inference moves one step further, when also trying to find the causality.

% Move 2 - Present research/purpose
In this paper, causal inference is investigated on a dataset collected onboard the double ended ferry Uraniborg.
A trend with high correlation between thruster utilization (TU) and fuel consumption (FC) is investigated to see if there is a direct causality, which can then be used to optimize the FC.

% Move 3 - Methods/materials/subjects/procedures
The inference was conducted in a controlled experiment onboard the ship.
% Move 4 - Results/findings
The experiment showed that there seems to be a direct causal relationship between the TU and FC.
An optimized TU, where only the aft thruster is utilized, is estimated to have between 9 and 17\% lower FC compared to the previous operation of the ship.

% Move 5 - Discussion/conclusion/significance
Conducting a controlled experiment to investigate if observed correlation from analysis of ship operational data is a good way for causal inference. Involving the crew in the experiment also had the beneficial side effect that they could immediately see the effect of the proposed changes, which might increase the chances to get a permanent change in the operations of the ship and thus a permanent reduction of the ships's FC. 





