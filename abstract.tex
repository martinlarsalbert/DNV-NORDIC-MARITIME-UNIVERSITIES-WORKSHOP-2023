%Purpose and scope
A relation between thruster allocation and fuel consumption with high correlation have been found onboard the double ended ferry Uraniborg. 
This correlation does however not imply causation. It can therefore not be concluded if the discovered thrust allocation fuel relations can be used to optimize the fuel consumption, until the causation, the cause and effect of this relationship has been determined.

%Methods
Blinded experiments were carried out to determine the causation for the present case. Other ways to determine the causation was also investigated such as system identification with well established mathematical models for ship dynamcis, Monte Carlo simulations and Markov chain analysis. These methods can be used in situations when the blinded experiments can not be conducted. 

%Results
The blinded experiments showed that there is a direct causal relationship between the thruster allocation and fuel consumption for the double ended ferry Uraniborg so that about 15\% fuel can be saved when running only on the aft thruster, compared to previous normal operation with both thrusters.  

%Conclusion
Conducting a blinded experiment to confirm that observed correlations from analysis of ship operational data is a very good way to confirm the causation. A beneficial side effect was that the crew involved in the experiment could immediately see the benefits of the optimized way to operate the ship. This makes it much easier to convince the crew to change the way they operate the ship and hopefully it will also increase their motivation to maintain this change to get a permanent reduction of the ship's fuel consumption.   